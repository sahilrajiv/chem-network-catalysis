\begin{tikzpicture}[remember picture, scale=\modGraphScale, baseline={([yshift={-0.5ex}]current bounding box)}, solid]
% dummy
\coordinate[overlay] (v-coord-0) at (2.500000, 0.866025) {};
\coordinate[overlay] (v-coord-1) at (2.000000, -0.000000) {};
\coordinate[overlay] (v-coord-2) at (1.000000, -0.000000) {};
\coordinate[overlay] (v-coord-3) at (1.000000, -1.000000) {};
\coordinate[overlay] (v-coord-4) at (2.500000, -0.866025) {};
\coordinate[overlay] (v-coord-5) at (0.000000, 0.000000) {};
\coordinate[overlay] (v-coord-8) at (1.000000, 1.000000) {};
\coordinate[overlay] (v-coord-9) at (0.000000, 1.000000) {};

\node[modStyleGraphVertex, at=(\modIdPrefix v-coord-0)] (\modIdPrefix v-0) {C};
\node[modStyleGraphVertex, at=(\modIdPrefix v-coord-1)] (\modIdPrefix v-1) {C};
\node[modStyleGraphVertex, at=(\modIdPrefix v-coord-2)] (\modIdPrefix v-2) {C};
\node[modStyleGraphVertex, at=(\modIdPrefix v-2.north), anchor=south, yshift=1pt] (\modIdPrefix v-2-aux) {H};
\node[modStyleGraphVertex, at=(\modIdPrefix v-2-aux)] {\phantom{H}};
\node[modStyleGraphVertex, at=(\modIdPrefix v-coord-3)] (\modIdPrefix v-3) {C};
\node[modStyleGraphVertex, text=red, at=(\modIdPrefix v-coord-4)] (\modIdPrefix v-4) {O};
\node[modStyleGraphVertex, text=red, at=(\modIdPrefix v-coord-5)] (\modIdPrefix v-5) {O};
\node[modStyleGraphVertex, text=red, at=(\modIdPrefix v-5.west), anchor=east] (\modIdPrefix v-5-aux) {H};
\node[modStyleGraphVertex, at=(\modIdPrefix v-5-aux)] {\phantom{H}};
\node[modStyleGraphVertex, text=blue, at=(\modIdPrefix v-coord-8)] (\modIdPrefix v-8) {N};
\node[modStyleGraphVertex, text=gray, at=(\modIdPrefix v-coord-9)] (\modIdPrefix v-9) {H};
\modDrawSingleBond{0}{1}{1}{1}{}{}
\modDrawSingleBond{1}{2}{1}{1}{}{}
\modDrawDoubleBond{1}{4}{1}{1}{}{}
\modDrawSingleBond{2}{3}{1}{1}{}{}
\modDrawSingleBond{2}{5}{1}{1}{NavyBlue, text=NavyBlue}{}
\modDrawSingleBond{8}{9}{1}{1}{NavyBlue, text=NavyBlue}{}
\end{tikzpicture}
